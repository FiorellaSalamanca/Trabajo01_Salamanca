\section{Actividad No 02 – Sistema de Control de Versiones Libres} 
Entre los sistemas de control de versiones que estan disponibles de manera libre tenemos los siguientes:

\begin{enumerate}[1.]

\item \textbf{CVS:} \\

El Concurrent Versions System (también conocido como Concurrent Versioning System) es una aplicación informática bajo licencia GPL que implementa un sistema de control de versiones.

Se encarga de mantener el registro de todo el trabajo y los cambios en los ficheros (código fuente principalmente) que conforman un proyecto y permite que distintos desarrolladores (potencialmente situados a gran distancia) colaboren.\\

\item \textbf{Subversion:} \\

SVN es conocido así por ser el nombre del cliente, el software en sí es llamado SUBVERSION.

SUBVERSION es un sistema centralizado para compartir información que fue diseñado como reemplazo de CVS.

La parte principal de SUBVERSION es el repositorio, el cual es un almacén central de datos. El repositorio guarda información en forma de árbol de archivos. .\\

\item \textbf{SVK:} \\

Aunque se ha construido sobre Subversion, probablemente SVK se parece más a algunos de los anteriores sistemas descentralizados que a Subversión. 

SVK soporta desarrollo distribuido, cambios locales, mezcla sofisticada de cambios, y la habilidad de "reflejar/clonar" árboles desde sistemas de control de versiones que no son SVK.\\

\item \textbf{Mercurial:} \\

MERCURIAL es otro sistema de control de versiones muy extendido, el creador y desarrollador principal de MERCURIAL es Matt Mackall. Está implementado principalmente haciendo uso del lenguaje de programación Python, pero incluye una implementación binaria de diff escrita en C.\\

\item \textbf{GIT:} \\

GIT es uno de los sistemas de control de versiones distribuidos más usados. Fue creado por Linus Torvalds1 y buscaba un sistema que cumpliera 4 requisitos básicos:

-No parecido a CVS
-Distribuido
-Seguridad frente a corrupción, accidental o intencionada
-Gran rendimiento en las operaciones

GIT está escrito en C y en gran parte fue construido para trabajar en el kernel de Linux, lo que quiere decir que desde el primer día ha tenido que mover de manera efectiva repositorios de gran tamaño.\\

\item \textbf{Bazaar:} \\

Bazaar está todavía en desarrollo. Será una implementación del protocolo GNU Arch, mantendrá compatibilidad con el procotolo GNU Arch a medida que evolucione, y trabajará con el proceso de la comunidad GNU Arch para cualquier cambio de protocolo que fuera requerido a favor del agrado del usuario.\\

\end{enumerate}


